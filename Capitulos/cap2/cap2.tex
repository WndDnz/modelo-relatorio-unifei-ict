\chapter{Desenvolvimento}\label{cap:desenvolvimento}

Apesar deste capítulo ser intitulado ``Desenvolvimento'', o conteúdo específico pode variar dependendo
da natureza do trabalho. Em trabalhos acadêmicos como monografias e teses, o segundo capítulo é
dedicado à apresentação detalhada do tema abordado, incluindo destacadamente a fundamentação
teórica e o estado da arte. Os outros capítulos geralmente tratam da metodologia aplicada, dos
experimentos realizados e da análise dos resultados obtidos, finalizando com um capítulo final
dedicado às conclusões elaboradas pelos autores e sugestões de trabalhos futuros e melhoramentos.

Neste modelo, abordaremos neste capítulo orientações para o uso adequado do estilo de formatação,
para que seu trabalho esteja em conformidade com as normas estabelecidas e seja visualmente
atraente. Afinal, um trabalho bem formatado facilita a leitura e compreensão por parte dos
avaliadores além de transmitir profissionalismo e enriquecer a experiência do leitor.

\section{Estruturação do conteúdo}\label{sec:estrutura-desenvolvimento}

Este modelo  foi estruturado com os capítulos separados em arquivos distintos, localizados na
pasta \texttt{Capitulos/}. Cada capítulo ocupa seu subdiretório próprio, o que facilita a
organização do conteúdo e permite que você trabalhe de forma modular. Para
incluir um novo capítulo, crie um arquivo \LaTeX\ na pasta correspondente e importe-o no arquivo
principal. Os nomes dos arquivos são livres, mas é recomendável utilizar nomes descritivos que
facilitem a identificação do conteúdo de cada capítulo ou sua sequência numérica, como utilizado
neste modelo.

O arquivo principal do relatório, \texttt{modelo-relatorio.tex}, importa os capítulos conforme
necessário, utilizando o comando \verb|\subimport| do pacote \texttt{import}. Assim, para economizar
no tempo de compilação durante a edição, você pode comentar temporariamente a importação de
capítulos que não estão sendo editados no momento. Uma vez que seu documento esteja pronto para a
compilação final, basta descomentar todas as importações e executar a compilação completa. Para mais
controle sobre a compilação de documentos \LaTeX, considere utilizar ferramentas como
\texttt{latexmk} ou \texttt{arara}, que automatizam o processo e garantem que todas as referências
e elementos do documento estejam atualizados corretamente. Este modelo foi desenvolvido para ser
utilizado com ambientes de compilação modernos, notadamente \texttt{latexmk} com \texttt{xelatex} ou
\texttt{lualatex}. Adaptações podem ser necessárias para o uso com outros ambientes.

\section{Estruturação de seções e subseções}\label{sec:secoes-desenvolvimento}

Quando se escreve um documento organizado, é importante estruturar o conteúdo em seções e subseções
de forma clara e lógica. Isso facilita a leitura e compreensão do texto, além de ajudar o leitor a
navegar pelo documento. Lembre-se que seu trabalho também servirá como referência para futuros
alunos e pesquisadores, portanto, uma boa estruturação é essencial.

Este modelo é baseado na classe \texttt{book}, que organiza o conteúdo em uma estrutura hierárquica
de capítulos, seções e subseções. A seguir, apresentamos algumas diretrizes para a estruturação
adequada do conteúdo:

\begin{itemize}
    \item \textbf{Capítulos}: Cada capítulo deve abordar um tema específico do trabalho. Utilize o
          comando \verb|\chapter{}| para criar novos capítulos. Lembre-se de que os capítulos são numerados
          automaticamente. Acrescente um rótulo ao capítulo para facilitar autorrefêrencias dentro do
          documento, usando o comando \verb|\label{}| logo após o título do capítulo. Por convenção, é
          interessante nomear os rótulos dos capítulos com o prefixo \texttt{cap:}, seguido de uma
          palavra-chave relacionada ao conteúdo do capítulo (por exemplo, \texttt{cap:introducao} para o
          capítulo de introdução).
    \item \textbf{Seções}: Dentro de cada capítulo, utilize o comando \verb|\section{}| para criar
          seções que dividem o conteúdo em partes menores e mais gerenciáveis. As seções também são
          numeradas automaticamente. Use o prefixo \texttt{sec:} para os rótulos das seções (por exemplo,
          \texttt{sec:objetivos} para a seção de objetivos).
    \item \textbf{Subseções}: Caso seja necessário dividir ainda mais o conteúdo dentro de uma seção,
          utilize o comando \verb|\subsection{}|. Isso ajuda a organizar o texto em tópicos específicos e
          facilita a leitura. Como prefixo para os rótulos das subseções, utilize \texttt{ssc:} (por exemplo,
          \texttt{ssc:sua-subseção} para a subseção de metodologia).
    \item \textbf{Subsubseções}: Para uma divisão ainda mais detalhada, é possível utilizar o comando
          \verb|\subsubsection{}|. Isso é útil para tópicos que exigem uma explicação mais aprofundada. Use o
          prefixo \texttt{sss:} para
          os rótulos das subsubseções (por exemplo, \texttt{sss:detalhes-tecnicos} para uma subsubseção
          chamada detalhes técnicos). Normalmente, é raro o uso de subsubseções, já que o nível de detalhamento
          excessivo pode dificultar a leitura do texto. Use com parcimônia.
    \item \textbf{Consistência}: Mantenha uma estrutura consistente ao longo do documento. Use os mesmos
          níveis de seções e subseções para temas semelhantes, garantindo que o leitor possa seguir o fluxo
          do texto facilmente.
\end{itemize}

\section{Citações}

Nenhum trabalho acadêmico está realmente completo sem uma discussão adequada sobre as fontes que o
sustentam. Sendo assim, citações de trabalhos da literatura associada ao tema são fundamentais para
que o trabalho tenha credibilidade e rigor acadêmico. As citações em texto devem seguir as normas da
Associação Brasileira de Normas Técnicas (ABNT), especificamente a NBR
6023:2018~\cite{nbr6023:2018}, que estabelecem diretrizes claras para a formatação
e apresentação das referências bibliográficas. Este modelo utiliza o pacote \texttt{biblatex}%
%
\footnote{Este modelo descreve de forma muito sucinta o uso do pacote \texttt{biblatex}.
    Recomenda-se a consulta à documentação do pacote para completo entendimento de como utilizá-lo}%
%
para gerenciar as referências, usando o pacote de estilo \texttt{abnt}, o que facilita a inclusão e
formatação das citações de acordo com essas normas. Sua bibliografia deve estar organizada em um
arquivo \texttt{.bib}, que neste modelo é denominado \texttt{referencias.bib} e está localizado na
raiz do projeto. \textbf{Atenção!}  O arquivo incluído é fictício, usando referências falsas apenas
para servir de exemplo de como organizar suas fontes.

Você também pode adicionar a bibliografia diretamente no arquivo principal do
documento, embora isso não seja recomendado para trabalhos mais extensos. Existem programas
gerenciadores de referências, como o \texttt{Zotero} e o \texttt{Mendeley}, que são úteis para
organizar suas fontes e são capazes de exportar suas referências diretamente em para um arquivo
\texttt{.bib}, facilitando a integração com o \LaTeX. Caso use uma dessas ferramentas, certifique-se
de revisar o arquivo exportado para garantir que todas as informações estejam corretas e completas e
simplesmente substiua o arquivo \texttt{referencias.bib} deste modelo pelo seu arquivo exportado.
Lembre-se de importá-lo no preâmbulo do arquivo principal do documento ou apenas o renomeie para
\texttt{referencias.bib}, substituindo o arquivo original. A seguir, alguns exemplos de como fazer citações corretamente no texto.

O primeiro tipo de citação é a citação direta, que consiste em transcrever exatamente as palavras de
um autor. Para isso, utilize o comando \texttt{\textbackslash cite} para incluir a referência no
texto. Citações diretas com até três linhas devem ser incorporadas ao parágrafo, entre aspas. Por
exemplo, segundo \textcite{fernandes2019}, ``a utilização de modelos padronizados em relatórios
técnicos é essencial para garantir a clareza e a uniformidade na apresentação dos resultados''. Note
que em \LaTeX, as aspas iniciais são representadas por dois acentos graves (\texttt{``}) e as aspas
finais por dois apóstrofos (\texttt{''}). Aspas simples são representadas por um acento grave
(\texttt{`}) e um apóstrofo (\texttt{'}) para abrir e fechar, respectivamente. Use aspas simples
para uma citação dentro de outra citação.

Citações diretas com mais de três linhas devem ser formatadas em um bloco separado, com recuo na
margem esquerda, sem apas e com a mesma tipografia do corpo do texto, porém em tamanho menor.
Supressões (quando você inicia a citação no meio de uma frase) devem ser indicadas por reticências entre colchetes
(\texttt{[\ldots]}) e acréscimos (quando você insere palavras para esclarecer o sentido da citação)
devem ser indicados por colchetes simples (\texttt{[texto adicional]}). Este modelo fornece o
comando \texttt{\textbackslash quote} para facilitar a formatação de citações longas. Veja o exemplo
a seguir:

\quote{martins2021}{%
[\ldots] modelos padronizados para elaboração de relatórios técnicos são fundamentais para a comunicação
científica eficaz. A estruturação comum garante que as ideias e fatos sejam devidamente comunicadas
aos receptores. A organização adequada do texto também é essencial para sua catalogação,
possibilitando a futuros leitores a compreenssão [grafia errada no original] do conteúdo de forma mais ágil e eficiente, assim
como sua consulta posterior.
}

Note que citações diretas longas não devem ser usadas com frequência excessiva, para evitar que o
texto se torne fragmentado e difícil de ler. Use-as apenas quando for realmente necessário
reproduzir fielmente as ideias do autor.

Sendo assim, o tipo mais comum de citação é a citação indireta,
que consiste em parafrasear as ideias de um autor com suas próprias palavras. Sintetize o conteúdo e
apresente-o de forma clara e concisa, sempre respeitando e atribuindo a fonte original.
Para fazer uma citação indireta, utilize o comando \texttt{\textbackslash textcite} ou
\texttt{\textbackslash parencite}, dependendo do contexto. Por exemplo, \textcite{silva2020} destaca a importância de
utilizar fontes confiáveis e atualizadas para embasar as argumentações em trabalhos acadêmicos.
Alternativamente, pode-se dizer que a utilização de fontes confiáveis é crucial para a credibilidade
do trabalho (cf. \parencite{silva2020}).

Existe ainda a \emph{citação de citação}, que ocorre quando você deseja citar uma fonte que foi
mencionada em outra fonte. Neste caso, usa-se a palavra ``apud'' para indicar que a citação foi
retirada de uma fonte secundária. No entanto, é importante ressaltar que a citação de citação deve
ser evitada sempre que possível, dando preferência à consulta direta da fonte original. Caso seja
necessário utilizar uma citação de citação, faça-o da seguinte forma: Segundo \textcite[p. 45]{oliveira2018}, a padronização de relatórios técnicos é essencial para garantir a clareza e a uniformidade na apresentação dos resultados (apud \textcite{fernandes2019}).