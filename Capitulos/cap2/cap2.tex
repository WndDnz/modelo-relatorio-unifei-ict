% cSpell:locale en,pt-BR - Adicionado para suporte ao corretor ortográfico
% Extensão Code Spell Checker no VS Code (muito útil!)
\chapter{Desenvolvimento}\label{cap:desenvolvimento}

Apesar deste capítulo ser intitulado ``Desenvolvimento'', o conteúdo específico pode variar dependendo
da natureza do trabalho. Em trabalhos acadêmicos como monografias e teses, o segundo capítulo é
dedicado à apresentação detalhada do tema abordado, incluindo destacadamente a fundamentação
teórica e o estado da arte. Os outros capítulos geralmente tratam da metodologia aplicada, dos
experimentos realizados e da análise dos resultados obtidos, finalizando com um capítulo final
dedicado às conclusões elaboradas pelos autores e sugestões de trabalhos futuros e melhoramentos.

Neste modelo, abordaremos neste capítulo orientações para o uso adequado do estilo de formatação,
para que seu trabalho esteja em conformidade com as normas estabelecidas e seja visualmente
atraente. Afinal, um trabalho bem formatado facilita a leitura e compreensão por parte dos
avaliadores além de transmitir profissionalismo e enriquecer a experiência do leitor.

\section{Estruturação do conteúdo}\label{sec:estrutura-desenvolvimento}

Este modelo  foi estruturado com os capítulos separados em arquivos distintos, localizados na
pasta \texttt{Capitulos/}. Cada capítulo ocupa seu subdiretório próprio, o que facilita a
organização do conteúdo e permite que você trabalhe de forma modular. Para
incluir um novo capítulo, crie um arquivo \LaTeX\ na pasta correspondente e importe-o no arquivo
principal. Os nomes dos arquivos são livres, mas é recomendável utilizar nomes descritivos que
facilitem a identificação do conteúdo de cada capítulo ou sua sequência numérica, como utilizado
neste modelo.

O arquivo principal do relatório, \texttt{modelo-relatorio.tex}, importa os capítulos conforme
necessário, utilizando o comando \verb|\subimport| do pacote \texttt{import}. Assim, para economizar
no tempo de compilação durante a edição, você pode comentar temporariamente a importação de
capítulos que não estão sendo editados no momento. Uma vez que seu documento esteja pronto para a
compilação final, basta descomentar todas as importações e executar a compilação completa. Para mais
controle sobre a compilação de documentos \LaTeX, considere utilizar ferramentas como
\texttt{latexmk} ou \texttt{arara}, que automatizam o processo e garantem que todas as referências
e elementos do documento estejam atualizados corretamente. Este modelo foi desenvolvido para ser
utilizado com ambientes de compilação modernos, notadamente \texttt{latexmk} com \texttt{xelatex} ou
\texttt{lualatex}. Ele foi construído usando pacotes e fontes disponíveis na instalação completa do
\LaTeX\ usando a distribuição TeX Live (\url{https://www.tug.org/texlive/}). Adaptações podem ser necessárias para o uso com outros ambientes.

\section{Estruturação de seções e subseções}\label{sec:secoes-desenvolvimento}

Quando se escreve um documento organizado, é importante estruturar o conteúdo em seções e subseções
de forma clara e lógica. Isso facilita a leitura e compreensão do texto, além de ajudar o leitor a
navegar pelo documento. Lembre-se que seu trabalho também servirá como referência para futuros
alunos e pesquisadores, portanto, uma boa estruturação é essencial.

Este modelo é baseado na classe \texttt{book}, que organiza o conteúdo em uma estrutura hierárquica
de capítulos, seções e subseções. A seguir, apresentamos algumas diretrizes para a estruturação
adequada do conteúdo:

\begin{itemize}
    \item \textbf{Capítulos}: Cada capítulo deve abordar um tema específico do trabalho. Utilize o
          comando \verb|\chapter{}| para criar novos capítulos. Lembre-se de que os capítulos são numerados
          automaticamente. Acrescente um rótulo ao capítulo para facilitar autorreferências dentro do
          documento, usando o comando \verb|\label{}| logo após o título do capítulo. Por convenção, é
          interessante nomear os rótulos dos capítulos com o prefixo \texttt{cap:}, seguido de uma
          palavra-chave relacionada ao conteúdo do capítulo (por exemplo, \texttt{cap:introducao} para o
          capítulo de introdução).
    \item \textbf{Seções}: Dentro de cada capítulo, utilize o comando \verb|\section{}| para criar
          seções que dividem o conteúdo em partes menores e mais gerenciáveis. As seções também são
          numeradas automaticamente. Use o prefixo \texttt{sec:} para os rótulos das seções (por exemplo,
          \texttt{sec:objetivos} para a seção de objetivos).
    \item \textbf{Subseções}: Caso seja necessário dividir ainda mais o conteúdo dentro de uma seção,
          utilize o comando \verb|\subsection{}|. Isso ajuda a organizar o texto em tópicos específicos e
          facilita a leitura. Como prefixo para os rótulos das subseções, utilize \texttt{ssc:} (por exemplo,
          \texttt{ssc:sua-subseção} para a subseção de metodologia).
\item \textbf{Subsubseções}: Para uma divisão ainda mais detalhada, é possível utilizar o comando \linebreak
          \verb|\subsubsection{}|. Isso é útil para tópicos que exigem uma explicação mais aprofundada. Use o
          prefixo \texttt{sss:} para
          os rótulos das subsubseções (por exemplo, \texttt{sss:detalhes-tecnicos} para uma subsubseção
          chamada detalhes técnicos). Normalmente, é raro o uso de subsubseções, já que o nível de detalhamento
          excessivo pode dificultar a leitura do texto. Use com parcimônia.
    \item \textbf{Consistência}: Mantenha uma estrutura consistente ao longo do documento. Use os mesmos
          níveis de seções e subseções para temas semelhantes, garantindo que o leitor possa seguir o fluxo
          do texto facilmente.
\end{itemize}

Na sequência, veja como se parecem as subseções e subsubseções.

\subsection{Subseção de exemplo}\label{ssc:exemplo}

Esta é uma subseção de exemplo. Note que as subseções são numeradas automaticamente e podem ser
referenciadas no texto usando o comando \verb|\ref{}| com o rótulo correspondente. Este modelo
também fornece o comando \verb"\refcomp{}{}" para incluir o nome do tipo de elemento referenciado
(Capítulo, Seção, Subseção etc. Note que sempre são feitas em maiúsculas) antes do número da
referência. Por exemplo, veja a seguir a referência para a subseção atual:
\refcomp{Subseção}{ssc:exemplo}. Note a saída formata e o \emph{hyperlink} criado automaticamente
para o elemento referenciado. O comando \verb"\ref{}" padrão do \LaTeX\ também funciona normalmente.

\subsubsection{Subsubseção de exemplo}\label{sss:exemplo}

Assim como as seções e subseções, as subsubseções também são numeradas automaticamente e podem ser
referenciadas no texto usando o comando \verb|\ref{}| com o rótulo correspondente. Por exemplo, veja
a seguir a referência para a subsubseção atual: \refcomp{Subsubseção}{sss:exemplo}.

\section{Citações}

Nenhum trabalho acadêmico está realmente completo sem uma discussão adequada sobre as fontes que o
sustentam. Sendo assim, citações de trabalhos da literatura associada ao tema são fundamentais para
que o trabalho tenha credibilidade e rigor acadêmico. As citações em texto devem seguir as normas da
Associação Brasileira de Normas Técnicas (ABNT), especificamente a NBR
6023:2018~\cite{nbr6023:2018}, que estabelecem diretrizes claras para a formatação
e apresentação das referências bibliográficas. Este modelo utiliza o pacote \texttt{biblatex}%
%
\footnote{Este modelo descreve de forma muito sucinta o uso do pacote \texttt{biblatex}.
    Recomenda-se a consulta à documentação do pacote para completo entendimento de como utilizá-lo}%
%
 para gerenciar as referências, usando o pacote de estilo \texttt{abnt}, o que facilita a inclusão e
formatação das citações de acordo com essas normas. Sua bibliografia deve estar organizada em um
arquivo \texttt{.bib}, que neste modelo é denominado \texttt{referencias.bib} e está localizado na
raiz do projeto. \textbf{Atenção!}  O arquivo incluído é fictício, usando referências falsas apenas
para servir de exemplo de como organizar suas fontes.

Você também pode adicionar a bibliografia diretamente no arquivo principal do
documento, embora isso não seja recomendado para trabalhos mais extensos. Existem programas
gerenciadores de referências, como o \texttt{Zotero} e o \texttt{Mendeley}, que são úteis para
organizar suas fontes e são capazes de exportar suas referências diretamente em para um arquivo
\texttt{.bib}, facilitando a integração com o \LaTeX. Caso use uma dessas ferramentas, certifique-se
de revisar o arquivo exportado para garantir que todas as informações estejam corretas e completas e
simplesmente substitua o arquivo \texttt{referencias.bib} deste modelo pelo seu arquivo exportado.
Lembre-se de importá-lo no preâmbulo do arquivo principal do documento ou apenas o renomeie para
\texttt{referencias.bib}, substituindo o arquivo original. A seguir, alguns exemplos de como fazer citações corretamente no texto.

O primeiro tipo de citação é a citação direta, que consiste em transcrever exatamente as palavras de
um autor. Para isso, utilize o comando \texttt{\textbackslash cite} para incluir a referência no
texto. Citações diretas com até três linhas devem ser incorporadas ao parágrafo, entre aspas. Por
exemplo, segundo \textcite{fernandes2019}, ``a utilização de modelos padronizados em relatórios
técnicos é essencial para garantir a clareza e a uniformidade na apresentação dos resultados''. Note
que em \LaTeX, as aspas iniciais são representadas por dois acentos graves (\texttt{``}) e as aspas
finais por dois apóstrofos (\texttt{''}). Aspas simples são representadas por um acento grave
(\texttt{`}) e um apóstrofo (\texttt{'}) para abrir e fechar, respectivamente. Use aspas simples
para uma citação dentro de outra citação.

Citações diretas com mais de três linhas devem ser formatadas em um bloco separado, com recuo na
margem esquerda, sem apas e com a mesma tipografia do corpo do texto, porém em tamanho menor.
Supressões (quando você inicia a citação no meio de uma frase) devem ser indicadas por reticências entre colchetes
(\texttt{[\ldots]}) e acréscimos (quando você insere palavras para esclarecer o sentido da citação)
devem ser indicados por colchetes simples (\texttt{[texto adicional]}). Este modelo fornece o
comando \texttt{\textbackslash quote{}{}} para facilitar a formatação de citações longas. Veja o exemplo
a seguir:

\quote{martins2021}{%
[\ldots] modelos padronizados para elaboração de relatórios técnicos são fundamentais para a comunicação
científica eficaz. A estruturação comum garante que as ideias e fatos sejam devidamente comunicadas
aos receptores. A organização adequada do texto também é essencial para sua catalogação,
possibilitando a futuros leitores a compreenssão [grafia errada no original] do conteúdo de forma mais ágil e eficiente, assim
como sua consulta posterior.
}

Note que citações diretas longas não devem ser usadas com frequência excessiva, para evitar que o
texto se torne fragmentado e difícil de ler. Use-as apenas quando for realmente necessário
reproduzir fielmente as ideias do autor.

Sendo assim, o tipo mais comum de citação é a citação indireta,
que consiste em parafrasear as ideias de um autor com suas próprias palavras. Sintetize o conteúdo e
apresente-o de forma clara e concisa, sempre respeitando e atribuindo a fonte original.
Para fazer uma citação indireta, utilize o comando \texttt{\textbackslash textcite} ou
\texttt{\textbackslash parencite}, dependendo do contexto. Por exemplo, \textcite{silva2020} destaca a importância de
utilizar fontes confiáveis e atualizadas para embasar as argumentações em trabalhos acadêmicos.
Alternativamente, pode-se dizer que a utilização de fontes confiáveis é crucial para a credibilidade
do trabalho (cf. \parencite{silva2020}).

Existe ainda a \emph{citação de citação}, que ocorre quando você deseja citar uma fonte que foi
mencionada em outra fonte. Neste caso, usa-se a palavra ``apud'' para indicar que a citação foi
retirada de uma fonte secundária. No entanto, é importante ressaltar que a citação de citação deve
ser evitada sempre que possível, dando preferência à consulta direta da fonte original. Caso seja
necessário utilizar uma citação de citação, faça-o da seguinte forma: Segundo \textcite[p. 45]{rodrigues2017}, a padronização de relatórios técnicos é essencial para garantir a clareza e a uniformidade na apresentação dos resultados (apud \textcite{lima2022}).

\subsection{Exemplos de citações por tipo de referência}

A seguir apresentamos exemplos curtos, para cada tipo de referência incluída no arquivo
\texttt{referencias.bib}, mostrando uma citação indireta (paráfrase) e uma citação direta (trecho
transcrito) usando os comandos disponíveis neste modelo%
\footnote{Observação: os exemplos a seguir são fictícios e servem apenas para ilustração. As
    citações diretas usadas aqui tem menos de 3 linhas, são apenas exemplos. Lembre-se que você deve
    usá-las apenas quando a transcrição é superior à 3 linhas.}%
.

\begin{itemize}
    \item \textbf{Artigo (article)} --- Indireta: \textcite{article1} discute técnicas de amostragem que facilitam a inferência em redes complexas. Direta:\par
          \quote{article1}{A amostragem aleatória aplicada a topologias complexas reduz o viés nas estimativas de grau médio.}

    \item \textbf{Livro (book)} --- Indireta: Segundo \textcite{book1}, métodos estatísticos clássicos ainda são válidos em muitos problemas práticos. Direta:\par
          \quote{book1}{"Os testes de hipóteses constituem ferramenta central na análise de dados experimentais."}

    \item \textbf{Anais / Trabalhos em Evento (inproceedings)} --- Indireta: \textcite{inproceedings1} apresenta otimizações aplicadas ao transporte urbano. Direta:\par
          \quote{inproceedings1}{"Aplicar heurísticas de baixo custo pode reduzir congestionamentos sem investimentos pesados."}

    \item \textbf{Capítulo em livro (incollection)} --- Indireta: \textcite{incollection1} descreve modelos de previsão de demanda. Direta:\par
          \quote{incollection1}{"Modelos probabilísticos permitem estimativas robustas mesmo com dados esparsos."}

    \item \textbf{Capítulo específico (inbook)} --- Indireta: \textcite{inbook1} cobre controle de qualidade na produção. Direta:\par
          \quote{inbook1}{"O controle estatístico de processo reduz a variabilidade de produção quando bem aplicado."}

    \item \textbf{Tese/Doutorado (phdthesis)} --- Indireta: Em sua tese, \textcite{phdthesis1} explora métodos numéricos para dinâmica de fluidos. Direta:\par
          \quote{phdthesis1}{"A convergência do método implícito foi observada mesmo em malhas não-uniformes."}

    \item \textbf{Dissertação/Mestrado (mastersthesis)} --- Indireta: \textcite{mastersthesis1} avalia algoritmos de roteamento. Direta:\par
          \quote{mastersthesis1}{"Algoritmos baseados em demand-aware routing apresentam melhor desempenho sob cargas variáveis."}

    \item \textbf{Relatório Técnico (techreport)} --- Indireta: O relatório do \textcite{techreport1} sumariza aplicações de sensoriamento remoto. Direta:\par
          \quote{techreport1}{"Sensoriamento remoto é essencial para mapeamento ambiental em larga escala."}

    \item \textbf{Manual (manual)} --- Indireta: O \textcite{manual1} descreve procedimentos de uso do sistema. Direta:\par
          \quote{manual1}{"Siga o procedimento de inicialização para evitar perda de dados durante a atualização."}

    \item \textbf{Recurso online (online)} --- Indireta: Dados públicos do \textcite{online1} podem ser usados para análises temporais. Direta:\par
          \quote{online1}{"A série histórica de indicadores permite estudos comparativos entre regiões."}

    \item \textbf{Material diverso (misc)} --- Indireta: \textcite{misc1} apresenta notas de aula úteis para introdução às redes neurais. Direta:\par
          \quote{misc1}{"A normalização das entradas melhora a estabilidade do treinamento."}

    \item \textbf{Anais (proceedings)} --- Indireta: Os \textcite{proceedings1} reúnem trabalhos relevantes da área. Direta:\par
          \quote{proceedings1}{"Os anais deste simpósio apresentam avanços interdisciplinares em computação."}

    \item \textbf{Patente (patent)} --- Indireta: A patente \textcite{patent1} descreve processo industrial inovador. Direta:\par
          \quote{patent1}{"O método aumenta a resistência do compósito sem aumentar o custo de produção."}

    \item \textbf{Documento não publicado (unpublished)} --- Indireta: \textcite{unpublished1} traz resultados preliminares sobre microplásticos. Direta:\par
          \quote{unpublished1}{"Estudos iniciais indicam concentração elevada de partículas em amostras costeiras."}
\end{itemize}

\section{Figuras e Tabelas}\label{sec:figuras-tabelas}

Assim como as citações, figuras e tabelas são elementos onipresentes em trabalhos acadêmicos e
técnicos. Eles ajudam a ilustrar conceitos, apresentar dados e facilitar a compreensão do conteúdo.
Figuras e tabelas devem ser inseridas no texto próximo ao local onde são mencionadas pela primeira
vez, para facilitar a leitura e evitar que o leitor precise procurar por elas em outras partes do
documento. Além disso, é importante numerar e rotular corretamente cada figura e tabela, utilizando
o ambiente \texttt{figure} para figuras e \texttt{table} para tabelas. Isso garante que as referências
a esses elementos sejam atualizadas automaticamente, mesmo que a ordem ou o número de figuras e
tabelas mude durante o processo de edição.

A norma ABNT NBR 14724:224~\cite{nbr14724:2024} estabelece diretrizes específicas para a
formatação e apresentação de figuras e tabelas em trabalhos acadêmicos. De acordo com essa norma,
as figuras devem ser numeradas consecutivamente em algarismos arábicos (Figura 1, Figura 2, etc.) e
devem possuir uma legenda descritiva, localizada acima da figura. As tabelas também devem ser
numeradas consecutivamente em algarismos arábicos (Tabela 1, Tabela 2, etc.) e devem possuir uma
legenda descritiva, localizada acima da tabela. Além disso, é importante garantir que as figuras e
tabelas estejam devidamente referenciadas no texto. Este modelo oferece os comandos
\verb|\reffig{}| e \verb|\reftable{}| para autoreferências formatadas, bastando passar o rótulo da figura/tabela. Use os prefixos \texttt{fig:} para figuras (por exemplo, \texttt{fig:exemplo})
e \texttt{tab:} para tabelas (por exemplo, \texttt{tab:exemplo}) ao criar os rótulos. Também existem
os comandos \verb|\reffigcomp{}| e \verb|\reftablecomp{}| para citar uma figura junto com seu título. Esta não é uma prática padrão, mas pode ser interessante em alguns contextos.

Figuras e tabelas devem referenciar suas fontes também, utilizando o comando \verb|\fonte{}|
fornecido por este modelo, que posiciona a fonte de forma adequada conforme as normas ABNT.  Note
que a fonte deve ser apresentada em fonte menor que o corpo do texto e alinhada à esquerda, conforme
exemplificado a seguir. Caso a Figura ou Tabela sejam de autoria própria, utilize a expressão ``o
próprio autor'' ou ``elaboração própria'' como fonte.

Veja aqui uma referência à \reffig{fig:exemplo} e à \reftable{tab:exemplo}.

\begin{figure}[!htbp]
    \centering
    \caption{Figura de exemplo}
    \label{fig:exemplo}
    \caption*{Uma figura de exemplo, elaborada pelo próprio autor. Note o alinhamento da ``fonte'', abaixo da figura. Esta legenda adicional é opcional, usada quando é necessário fornecer uma descrição mais detalhada da figura, complementando a legenda principal.}
    \includegraphics[width=0.9\textwidth]{example-image}
    \fonte{o próprio autor}
\end{figure}

\begin{table}[!htbp]
    \centering
    \caption{Tabela de exemplo}
    \caption*{A legenda da tabela deve conter uma descrição clara e concisa do conteúdo apresentado.}
    \label{tab:exemplo}
    \begin{tabular}{@{}lllll@{}}
        \toprule
        Coluna 1 & Coluna 2 & Coluna 3 & Coluna 4 & Coluna 5 \\ \midrule
        1        & 2        & 2        & 1        & 3        \\
        1        & 2        & 1        & 4        & 3        \\
        5        & 3        & 4        & 3        & 5        \\ \bottomrule
    \end{tabular}
    \fonte{o próprio autor}
\end{table}

Usando o pacote \texttt{subcaption}, é possível incluir múltiplas figuras ou tabelas dentro de um
mesmo ambiente \texttt{figure} ou \texttt{table}, cada uma com sua própria legenda, além da legenda
principal. Veja a \reffig{fig:exemplo-multiplo} como exemplo.

\begin{figure}[!htbp]
    \centering
    \caption[Figura múltipla]{Exemplo de figura múltipla}
    \label{fig:exemplo-multiplo}
    \begin{subfigure}{0.45\textwidth}
        \centering
        \includegraphics[width=\textwidth]{example-image-a}
        \caption{Subfigura A}
        \label{fig:subfig-a}
    \end{subfigure}
    \begin{subfigure}{0.45\textwidth}
        \centering
        \includegraphics[width=\textwidth]{example-image-b}
        \caption{Subfigura B}
        \caption*{Cada subfigura pode ter sua própria legenda}
        \label{fig:subfig-b}
    \end{subfigure}
    \fonte{\cite{ibge2020,almeida2022}}
\end{figure}

Lembre-se que você pode sempre consultar a documentação do pacote \texttt{subcaption} para mais
detalhes sobre como utilizá-lo de forma eficaz. Ele pode ser encontrado no diretório oficial do
\LaTeX\, na \gls{ctan}, no endereço eletrônico (\url{https://ctan.org/pkg/subcaption}).

\section{Equações}\label{sec:equacoes}

Equações também são elementos comuns em trabalhos acadêmicos, especialmente em áreas como
matemática, física e engenharia. Equações podem ser apresentadas de duas formas, \emph{inline} ou
em blocos separados. Equações \emph{inline} são inseridas diretamente no texto, enquanto equações em
blocos separados são destacadas do texto, geralmente centralizadas e numeradas para referência. Veja
um exemplo de equação \emph{inline}, usando a equação considerada por muitos, como a mais bela de
todas, a \emph{Identidade de Euler}: \(e^{i\pi} + 1 = 0\). O \LaTeX\ provê uma maneira simples de
formatar equações matemáticas, com uma variedade de símbolos e operadores disponíveis. Para equações
em blocos separados, utilize o ambiente \texttt{equation}, como mostrado a seguir:

\begin{equation}
    a^2 + b^2 = c^2
    \label{eq:pitagoras}
\end{equation}

Equações em blocos também podem ser referenciadas no texto. O modelo provê os comandos
\verb|\refeq{}| e \verb|\refeqcomp{}| para referenciar equações com formatação consistente com o
estilo do documento. Por exemplo, veja a referência à \refeq{eq:pitagoras}. E a referência
completa à mesma: \refeqcomp{eq:pitagoras}{Equação do Teorema de Pitágoras}.

\section{Abreviaturas, siglas e símbolos}\label{sec:abreviaturas-siglas-simbolos}

Caso seu documento utilize muitas abreviaturas, siglas ou símbolos técnicos, é recomendável incluir
uma lista dedicada a esses elementos. Isso ajuda o leitor a entender rapidamente o significado de
termos técnicos ou abreviados usados ao longo do texto. Também serve como referência rápida, caso o
leitor não se lembre imediatamente do significado de uma sigla ou símbolo específico. A lista de
abreviaturas pode ser construída manualmente ou utilizando pacotes específicos do \LaTeX, como o
\texttt{nomencl} ou o \texttt{glossaries}. Este modelo inclui um arquivo de exemplo para a lista de
abreviaturas e siglas, localizado em \texttt{Preambulo/lista-abreviaturas.tex}. Para
incluir a lista no seu documento, descomente a linha correspondente no arquivo principal do
documento. Para imprimir a lista de abreviaturas e a lista de símbolos, descomente os comandos
\verb|\printglossary|, no arquivo principal do modelo. A lista de abreviaturas e a lista de símbolos
serão geradas automaticamente com base nas definições fornecidas no arquivo de abreviaturas e no
arquivo de símbolos.

Para referenciar uma abreviatura ou símbolo no texto, utilize o comando \verb|\gls{}|, passando como
argumento a chave definida para a abreviatura ou símbolo no arquivo de definições. Por exemplo, para
referenciar a abreviatura \emph{Mínimo Múltiplo Comum}, definida com a chave \texttt{mmc}, utilize o
comando \verb|\gls{mmc}|. Veja: \gls{mmc}. Na primeira vez que a abreviatura for referenciada, o
nome completo será exibido junto com a abreviatura entre parênteses. Nas referências subsequentes,
apenas a abreviatura será exibida, observe: \gls{mmc}. Isso garante que o leitor compreenda o
significado da abreviatura desde a primeira menção. A lista de abreviaturas e símbolos será
atualizada automaticamente com base nas referências feitas no texto, garantindo apenas abreviaturas
que foram citadas no texto apareçam na lista. Um outro ponto importante, é que o pacote
\texttt{glossaries} é integrado ao pacote \texttt{hyperref}, garantindo que todas as referências a
abreviaturas e símbolos sejam \emph{hyperlinks} clicáveis, facilitando a navegação no documento.

Símbolos funcionam de forma parecida, eles devem ser citados no texo para que apareçam na lista de
símbolos. Use o comando \verb|\gls{}| também para referenciar os símbolos, veja: \gls{symb:pi}. Caso
deseje imprimir a descrição do símbolo no texto, utilize o comando \verb|\glsdesc{}|, veja a
descrição para o símbolo \gls{symb:pi}: \glsdesc{symb:pi}. Se seu trabalho possui muitas equações
usando símbolos específicos, é uma boa prática mantê-los organizados em um arquivo de definições
separado, como o fornecido neste modelo em \texttt{Preambulo/lista-simbolos.tex}.

Um ponto importante: a lista de abreviaturas não é gerada automaticamente. Ao incluir uma nova
abreviatura ou símbolo no arquivo de definições, você deve compilar o documento para gerar o
documento auxiliar necessário para a lista de abreviaturas. Dependendo do ambiente de compilação que
você está utilizando, pode ser necessário executar comandos adicionais para gerar a lista de
abreviaturas corretamente. Consulte a documentação do pacote \texttt{glossaries}
(\url{https://www.ctan.org/pkg/glossaries}) para mais detalhes sobre como gerar a lista de
abreviaturas e a lista de símbolos. Geralmente, isto envolve uma chamada ao comando
\texttt{makeglossaries} passando o nome do arquivo principal na linha de comando.
