\chapter{Introdução}\label{cap:introducao}

Na Introdução, apresenta-se um resumo estendido do trabalho. Apresentam-se as motivações que levaram
o(s) autor(es) a escolher o tema, o problema de pesquisa, os objetivos do trabalho, a metodologia aplicada,
a estrutura do trabalho e uma breve descrição dos capítulos subsequentes.

A Introdução deve ser redigida de forma clara e objetiva, proporcionando ao leitor uma visão geral
do conteúdo do trabalho. Deve-se evitar detalhes excessivos, que serão abordados nos capítulos
seguintes. A Introdução deve ser escrita de forma a despertar o interesse do leitor pelo tema abordado.
Além disso, é importante que a Introdução esteja alinhada com o restante do trabalho, refletindo os
objetivos e a metodologia adotada.

A seguir, apresenta-se a estrutura sugerida para a Introdução:
\begin{itemize}
    \item Contextualização do tema: Os primeiros parágrafos da Introdução devem situar o leitor no contexto do tema abordado, apresentando motivações e justificativas para a escolha do tema. Pode também conter um resumo do estado da arte, destacando trabalhos relevantes na área.
    \item Problema de pesquisa e justificativas: Nos parágrafos seguintes, definir claramente o problema que será abordado no trabalho, destacando sua relevância e impacto na área de estudo. Justificar a importância do problema e sua contribuição para o avanço do conhecimento.
    \item Objetivos: Apresentar o objetivo geral e os objetivos específicos do trabalho em uma seção própria, indicando o que se pretende alcançar com a pesquisa
    \item Estrutura do trabalho: Apresentar a organização dos capítulos subsequentes, indicando o conteúdo abordado em cada um deles.
\end{itemize}

\section{Objetivos}\label{sec:objetivos}

O objetivo geral de seu trabalho deve estar destacado. É permitido fazer uma breve introdução de um ou dois parágrafos antes de apresentar o objetivo geral . Também é permitido destacá-lo utilizando ênfase (comando \verb"\emph{}"). Em seguida, devem ser listados os objetivos específicos do trabalho. O \LaTeX possui vários comandos para listas, como os ambientes \verb"itemize" e \verb"enumerate". Para este documento de exemplo, o objetivo geral é \emph{apresentar o modelo de relatório técnico-científico para os cursos do ICT}. Os objetivos específicos são:

\begin{itemize}
    \item \textbf{Fornecer diretrizes} para a formatação e estruturação de relatórios técnicos-científicos, alinhadas com as normas da Unifei ICT.
    \item \textbf{Facilitar o processo de redação} para estudantes e pesquisadores, oferecendo um modelo padronizado que pode ser facilmente adaptado às suas necessidades.
    \item \textbf{Promover a consistência} na apresentação de relatórios técnicos-científicos dentro do Instituto de Ciências Tecnológicas, garantindo que todos os documentos sigam um formato uniforme.
\end{itemize}

\section{Estrutura do trabalho}\label{sec:estrutura}

Este modelo de relatório segue uma estrutura padrão que pode ser adaptada conforme as necessidades específicas do trabalho. O \refcomp{Capítulo}{cap:introducao} apresenta a introdução ao tema, incluindo os objetivos e a estrutura do relatório. Os capítulos seguintes são referentes ao desenvolvimento do tema. Geralmente, apresentam a fundamentação teórica, a metodologia de pesquisa, os experimentos realizados e a análise dos resultados. Finalmente, o \refcomp{Capítulo}{cap:conclusao} traz as conclusões do trabalho, destacando as contribuições e sugestões para trabalhos futuros.


\section{Citações}
Exemplo de citação \cite{fernandes2019}.

\section{Seção de exemplo}
Texto de exemplo para a seção.

\subsection{Subseção de exemplo}
Texto de exemplo para a subseção.

\subsubsection{Subsubseção de exemplo}
Texto de exemplo para a subsubseção.

