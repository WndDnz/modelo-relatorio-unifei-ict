% cSpell:locale en,pt-BR - Adicionado para suporte ao corretor ortográfico
% Extensão Code Spell Checker no VS Code (muito útil!)
\chapter{Conclusão}\label{cap:conclusao}

Chegamos ao capítulo final do modelo de relatório. Claro, no seu relatório real, este capítulo deve
conter as conclusões do seu trabalho, resumindo os principais pontos abordados e destacando as
contribuições mais relevantes. Aqui, apenas ilustramos a estrutura do documento. Utilizamos o pacote
\texttt{lipsum} para gerar texto placeholder, como você pode ver a seguir.

\lipsum[6-7]

As instruções contidas neste modelo são apenas sugestões iniciais. Sinta-se à vontade para
adaptá-las conforme as necessidades específicas do seu relatório e as diretrizes da instituição.
O \LaTeX oferece uma flexibilidade enorme na formatação de documentos técnicos e científicos,
permitindo que você crie um relatório que atenda exatamente às suas expectativas. Este modelo visa
facilitar esse processo, fornecendo uma base sólida para o desenvolvimento do seu trabalho. No
entanto, sinta-se livre para personalizá-lo conforme necessário. Apenas tenha em mente as boas
práticas de redação técnica e científica, garantindo que seu relatório seja claro, conciso e bem
estruturado e que siga as normas da ABNT. Estudar o código fonte deste modelo pode ser uma ótima
forma de aprender mais sobre o \LaTeX e suas capacidades.

Desejamos sucesso na elaboração do seu relatório técnico-científico!